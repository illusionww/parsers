\documentclass[11pt, a4paper]{article}

\usepackage[a4paper, left = 10mm, right = 10mm, top = 10mm, bottom = 17mm, bindingoffset = 0cm, columnsep = 1cm]{geometry}
\usepackage[T2A]{fontenc}
\usepackage[utf8]{inputenc}
\usepackage[russian]{babel}
\usepackage{amsmath,amssymb,amsthm}
\usepackage{pscyr}
\usepackage{indentfirst}
\usepackage[pdftex,colorlinks,unicode,bookmarks]{hyperref}

\sloppy

\begin{document}

Опишу здесь разное, что касается 

\section*{Российская общественная инициатива \url{https://www.roi.ru/}}
Описание, что какая колонка обозначает (ну мало ли): 
\url{https://static.roi.ru/content/doc/roi-api-v1.0.pdf}

В том списке они забыли поле ``decision'', поле с текстом о предлагаемом решении.
Я забираю все поля, кроме внутренних id разных статусов, documents и authorship.
В остальном я тут был супераккуратен.

Этот сайт имеет очень статичное api. Здесь принудительная разбивка по статусам, и нет никакой возможности фильтровать по дате.

Самые сырые данные находятся в файлах:
\begin{itemize}
	\item all\_poll.csv, all\_poll.xlsx,
	\item all\_advisement.csv, all\_advisement.xls,
	\item all\_complete.csv, all\_complete.xlsx,
	\item all\_archive.csv, all\_archive.xlsx.
\end{itemize}
Совокупно они содержат вообще все записи РОИ.

Дальше я смержил их в один, отсортировал по date.poll.begin и отфильтровал по [1.02.2015:29.02.2016]

\end{document}