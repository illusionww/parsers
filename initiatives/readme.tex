\documentclass[11pt, a4paper]{article}

\usepackage[a4paper, left = 10mm, right = 10mm, top = 10mm, bottom = 17mm, bindingoffset = 0cm, columnsep = 1cm]{geometry}
\usepackage[T2A]{fontenc}
\usepackage[utf8]{inputenc}
\usepackage[russian]{babel}
\usepackage{amsmath,amssymb,amsthm}
\usepackage{pscyr}
\usepackage{indentfirst}
\usepackage[pdftex,colorlinks,unicode,bookmarks]{hyperref}

\sloppy

\begin{document}

\section*{Российская общественная инициатива \url{https://www.roi.ru/}}
Описание, что какая колонка обозначает (ну мало ли): 
\url{https://static.roi.ru/content/doc/roi-api-v1.0.pdf}

В том списке они забыли поле ``decision'', поле с текстом о предлагаемом решении.
Я забираю все поля, кроме внутренних id разных статусов, documents и authorship.
Почти все тексты написаны в несколько абзацев -- я всё это сохранил.

Этот сайт имеет очень статичное api. Здесь принудительная разбивка по статусам, и нет никакой возможности фильтровать по дате.

Самые сырые данные находятся в файлах:
\begin{itemize}
	\item poll.csv, poll.xlsx, 2477 записей
	\item advisement.csv, advisement.xls, 2 записи
	\item complete.csv, complete.xlsx, 21 запись
	\item archive.csv, archive.xlsx, 4947 записей
\end{itemize}
То есть всего $2477 + 2 + 21 + 4947 = 7447$, что совпадает с числом на главной сайта. Это хорошо.

Две петиции вылезли за ограничение длины ячейки в Excel (32767 символов) -- это петиции с id 14332 и 24800. И они обе об обеспечении информационно-психологической безопасности, что намекает. Я положил эти пасты отдельно в description\_14332.txt и description\_24800.txt -- потому что в таблицу они не влезают. В таблице -- обрезанные варианты.

Дальше я смержил все петиции в один файл -- all.xlsx, отсортировал по date.poll.begin и отфильтровал по [1.02.2015:29.02.2016] -- final\_roi.xlsx.
Осталось больше, чем было указано: тут 2625, а ты говорила про 2443. Может быть ты учитывала 12 месяцев, а тут сейчас охватывается 13 (февраль дважды).

\section*{Демократор.ру \url{https://democrator.ru/}}
Здесь есть несколько проблем:
\begin{itemize}
	\item Сайт не показывает точную дату, есть только обозначения ``n дней назад'', ``n месяцев назад'', ``n лет назад''.
	И даже в таком виде этой информации нет в каком-то определенном поле а-ля ``Петиция создана:'', а лежит в истории. Даты, которые я взял -- первая запись в этой истории, обычно это событие создания петиции. 
	Я засунул эту информацию в колонку ``raw\_start\_date'', а в следующей колонке руками посчитал дату создания. Для дат меньше месяца -- с точностью до дня, для дат меньше года -- с точностью до месяца, для дат больше года -- не стал писать ``2015'', но, я думаю, при желании это будет не сложно сделать.
	\item Предполагается, что этих петиций за выбранный промежуток там 64. Видимо, если отбросить петиции с датой ``2 года назад'' (их 12) и петиции за март (3), то получится как раз 80 - 12 - 3 = 65. Может, на момент подсчетов еще не настало 29 февраля, например, и их было 64. Ок, тогда в файле final\_democrator.xlsx будут только петиции нужного периода времени.
\end{itemize}

\section*{Change.org \url{https://change.org/}}
У них есть api для того, чтобы доставать петицию, но нет апи для поиска.
Так что я открыл вкладку последние и забрал все 744 страницы id, после чего запросил все 7432 петиции в api.
api отдает какие-то стремные заголовки -- например, ``ернатор : Сохранить на посту главного тренера ХК "Водник" Батова Олега Валерьевича.
'' вместо ``Сохранить на посту главного тренера ХК "Водник" Батова Олега Валерьевича.'', но иногда совсем промашки, например ``???????????? ??????? ?????? ? ?????? ?????-??????????'' или ``femojis-ru'' вместо ``Пришло время для \#фемоджи! (Нам нужна ваша помощь!)''. Так что я обошел все страницы и распарсил все заголовки отдельно (new\_title\_change.org.xlsx).

Снова есть длинные петиции: это 3121266, 4261744, 4308608 -- снова приложил их отдельно.

Отфильтрованные по дате данные лежат в final\_change.org.xlsx.

\end{document}